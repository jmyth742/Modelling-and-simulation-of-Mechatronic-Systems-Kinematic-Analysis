

\section{Methodology}

\subsection{Analysis of Geometries}
First looking at Figure \ref{fig:FrontView}, giving a cross sectional view in 2D (Y and Z planes) of the platform and the actuating mechanism. The geometries of the system become apparent when breaking down the components of the system. The system is comprised of a length, d, modelled as the mid point between $P_{11}$ and $P_{31}$, to begin with this length will be used as a simplified version of the platform base, this is discussed in more detail in the next section. The value r, the fixed length of the actuating arms at $q_{11}$ and $q_{31}$. The varying angles of $\theta_{1}$ and $\theta_{3}$, which constrain the configurations of the rest of the system, this is discussed in greater detail in the constraints section. Finally the fixed angles of the sliders $\alpha$, dictating the gradient of the slope for the sliders.  
\subsubsection{The Platform}
For purposes of simplicity in this initial 2D model, the platform is seen to be a line between $P_{11}$ and $P_{31}$, with mid point half way between the 2 points. This approach is used to identify the range of positions the platform edges can exist and if they are suitable. In reality and further into the development of the model the platform will be modelled as an elongated 3D trapezoidal shape, the reason for this is to achieve more dynamic range of motion from a lesser work space. This shape allows the base depending on the configuration of the sliders to act like more than one shape at once. Currently the motion of the platform is limited to rotation in the 'Z' plane and translation in both the 'Z' and the 'Y' plane. As the development of the model is continued, this will be expanded to take into consideration the movement in the 'X' axis also. 
\subsubsection{Sliding Joints}
When analyzing the sliders, it can be seen that from Figure \ref{fig:FrontView}, the mechanism can be viewed as a set of 2 dynamically changing triangle as a function of time. As the varying angles, $theta$ change over time, so too does the configuration of the platform base. Geometrically the sliders can only translate laterally in the Y plane (this is also true for the eventual 3D model) with 1 degree of freedom, constrained by the length of the runners they exist in. The arms, can translate laterally as a function of the the slider base and in the 'Z' plane as a function of $\theta$, constrained by the gradient of $l_{x}(t)$. 

\subsection{Creating Reference Frames}
Figure \ref{fig:FrontView} shows position and orientation of the system's reference frames. The left hand side is displayed in red as subgroup 1, whereas the right hand side is colored blue as subgroup 3. The notation is picked in a way that it's scale able to a 3D solution, where the left hand site would be expanded by a subgroup 2 and the right hand side would receive an additional subgroup 4.\newline
To be able to keep the angles positive and between 0$^{\circ}$ and 90$^{\circ}$, $RF_{12}$ and $RF_{31}$ have been rotated around their $Z$ axis by a factor of $\pi$. The translation along Y of the respective frame is given with the variable $q_{ij}\left( t \right)$.

\begin{align*}
    RF_{11} =  \left[ \begin {array}{cccc} 1&0&0&0\\ \noalign{\medskip}0&1&0&q_{11}\left( t \right)
\\ \noalign{\medskip}0&0&1&0\\ \noalign{\medskip}0&0&0&1\end {array}
 \right] 
  RF_{31} =\left[ \begin {array}{cccc} -1&0&0&0\\ \noalign{\medskip}0&-1&0&q_{31
} \left( t \right) \\ \noalign{\medskip}0&0&1&0\\ \noalign{\medskip}0&0
&0&1\end {array} \right]  \\
    RF_{32} =  \left[ \begin {array}{cccc} 1&0&0&0\\ \noalign{\medskip}0&1&0&q_{32}
 \left( t \right) \\ \noalign{\medskip}0&0&1&0\\ \noalign{\medskip}0&0
&0&1\end {array} \right]
 RF_{12} =  \left[ \begin {array}{cccc} -1&0&0&0\\ \noalign{\medskip}0&-1&0&q_{12
} \left( t \right) \\ \noalign{\medskip}0&0&1&0\\ \noalign{\medskip}0&0
&0&1\end {array} \right]
\end{align*}


\subsection{Creating Points}

 The points $P_{11}$ and $P_{31}$ are computed by using their respective reference frame, rotate it by an angle $\theta_i$ and translate it along it's new $Y$ axis by a positive value of $r$.\newline
 The points $P_{12}$ and $P_{32}$ are computed by using their respective reference frame, rotate it by an angle $\alpha$ and translate it along it's new $Y$ axis by a positive value of $l_i\left( t \right)$.

\begin{align*}
P_{ij} =  \left[ \begin {array}{c} 0\\ \noalign{\medskip}r\cos \left( \theta_{i
} \left( t \right)  \right) \\ \noalign{\medskip}r\sin \left( \theta_{
i} \left( t \right)  \right) \end {array} \right] \\
P_{ij} =  \left[ \begin {array}{c} 0\\ \noalign{\medskip}l_{i} \left( t
 \right) \cos \left( \alpha \right) \\ \noalign{\medskip}l_{i} \left( 
t \right) \sin \left( \alpha \right) \end {array} \right] 
\end{align*}

 Finally for creating the end effector $E$, a vector $\overset{\large\rightharpoonup}{\small{V}}$ is created that connects  $P_{11}$ and $P_{31}$ (or alternatively $P_{12}$ and $P_{32}$). $\overset{\large\rightharpoonup}{\small{V}}$ is devided by two and added to $P_{11}$ (eq. \ref{Endeffector}).
 Given the surge constraint of the system of 16$^{\circ}$ the angle may be between $\pm8^{\circ}$. Because the angle may be negative, and is guaranteed to be within [$\pi$,$\pi$], $\gamma$ can be computed using eq.\ref{Orientation}.
 
 \begin{equation}
    E_{position} = P_{11} + \frac{ \overset{\large\rightharpoonup}{\small{V}} }{2}
    \label{Endeffector}
 \end{equation}
 
  \begin{equation}
    \gamma = \arctan{ (\overset{\large\rightharpoonup}{\small{V}}_Z,\overset{\large\rightharpoonup}{\small{V}}_Y ) }
    \label{Orientation}
 \end{equation}

\subsection{Establishing Constraints}

The system currently is a 2D problem with 3 DOF. However several separate entities have been evaluated and need to be put into relation to another. Creating the constrain equations is necessary to create the kinematic model of the system.

\subsubsection{System Constraints}

The system constraints are the constraints that create the kinematic model of the system.

\begin{equation}
   \overline{ P_{i1}P_{i2} } = \overset{\large\rightharpoonup}{\small{0}}
\end{equation}
The points $P_{i1}$ and $P_{i2}$ are the same point

\begin{equation}
   ||\overline{ P_{1j}P_{3j} } || = d(q)
\end{equation}
The points $P_{1j}$ and $P_{i2}$ have a rigid connection that depends on the $X$ translation of the system. This translation can be neglected since this solution is focused on the 2D problem. However in regards to further investigation the distance between the points is given my a function $d(q)$, where $q$ represents the translation of the platform along the end effectos $x$ axis.

These constraints result in a set of system equations (eq.\ref{systemEQ}). The distance represented by the function $d(q)$ is assumed to be $\frac{3}{2}r$.

\begin{equation*}
\Phi_{1} = \left[ \begin {array}{c} \sqrt { \left( -l_{1} \left( t \right) \cos
 \left( \alpha \right) -l_{3} \left( t \right) \cos \left( \alpha
 \right) -q_{32} \left( t \right) +q_{12} \left( t \right)  \right) ^{
2}+ \left( -l_{1} \left( t \right) \sin \left( \alpha \right) +l_{3}
 \left( t \right) \sin \left( \alpha \right)  \right) ^{2}}-3/2\,r
\\ \noalign{\medskip}-r\cos \left( \theta_{1} \left( t \right) 
 \right) -l_{1} \left( t \right) \cos \left( \alpha \right) +q_{12}
 \left( t \right) -q_{11} \left( t \right) \\ \noalign{\medskip}-r\sin
 \left( \theta_{1} \left( t \right)  \right) +l_{1} \left( t \right) 
\sin \left( \alpha \right) \\ \noalign{\medskip}-r\cos \left( \theta_{
3} \left( t \right)  \right) -l_{3} \left( t \right) \cos \left( 
\alpha \right) -q_{32} \left( t \right) +q_{31} \left( t \right) 
\\ \noalign{\medskip}-r\sin \left( \theta_{3} \left( t \right) 
 \right) +l_{3} \left( t \right) \sin \left( \alpha \right) 
\end {array} \right] 
\label{systemEQ}
\end{equation*}

Given these constraints we can clearly identify 4 dependent variables in $l_1(t),l_3(2),\theta_1(t)$ and $\theta_3(t)$. However 4 variables remain represented by  $q_{ij}(t)$. Given that it is a 2D problem and there 5 constraint equations, one of the sliders will be dependent on the other three, leaving both 3 DOF and independent variables.

A further procedure would be to pick 3 independent coordinates of the $q$-set and solve $\Phi_1$. However since the system is to be analyzed with 2 out of 3 DOF fixed, further constraints will be added before solving.

\subsubsection{Position Constraints}

The 2D problem has 3 possible independent movements in lateral translation (along $Y$), longitudinal translation (along $Z$) and rotation (along $X$). The lateral translation will be ignored, since it is trivial behaviour, kinematic solely dependent on the length of the systems slides. This leaves the "Surge" and the "Heave" as two interesting subject of analysis.
To evaluate the surge of the system, the end effector position E will be held constant at an point P. The surge will be evaluated at an value of $h=333mm$, meaning the end effector will be held stable $33cm$ above the ground.

\begin{align*}
    P =  \left[ \begin {array}{c} 0\\ \noalign{\medskip}h\end {array} \right]
\end{align*}{}

This will result in two more constraint equation, leaving only one independent parameter to adjust. However in order to analyzee the angle of rotation, an additional variable is introduced $theta_2(t)$. This will represent the $gamma$ value and the resulting surge value of the system.

The new $\Phi$ vector consists of 8 equations, 8 dependent variables $Q_D$ and one independent variable $Q_I$.

\begin{equation*}
 \Phi =   \left[ \begin {array}{c} \sqrt { \left( -l_{1} \left( t \right) \cos
 \left( \alpha \right) -l_{3} \left( t \right) \cos \left( \alpha
 \right) -q_{32} \left( t \right) +q_{12} \left( t \right)  \right) ^{
2}+ \left( -l_{1} \left( t \right) \sin \left( \alpha \right) +l_{3}
 \left( t \right) \sin \left( \alpha \right)  \right) ^{2}}-3/2\,r
\\ \noalign{\medskip}-r\cos \left( \theta_{1} \left( t \right) 
 \right) -l_{1} \left( t \right) \cos \left( \alpha \right) +q_{12}
 \left( t \right) -q_{11} \left( t \right) \\ \noalign{\medskip}-r\sin
 \left( \theta_{1} \left( t \right)  \right) +l_{1} \left( t \right) 
\sin \left( \alpha \right) \\ \noalign{\medskip}-r\cos \left( \theta_{
3} \left( t \right)  \right) -l_{3} \left( t \right) \cos \left( 
\alpha \right) -q_{32} \left( t \right) +q_{31} \left( t \right) 
\\ \noalign{\medskip}-r\sin \left( \theta_{3} \left( t \right) 
 \right) +l_{3} \left( t \right) \sin \left( \alpha \right) 
\\ \noalign{\medskip}1/2\,r\cos \left( \theta_{1} \left( t \right) 
 \right) -1/2\,r\cos \left( \theta_{3} \left( t \right)  \right) +1/2
\,q_{31} \left( t \right) +1/2\,q_{11} \left( t \right) 
\\ \noalign{\medskip}1/2\,r \left( \sin \left( \theta_{1} \left( t
 \right)  \right) +\sin \left( \theta_{3} \left( t \right)  \right) 
 \right) -h\\ \noalign{\medskip}\arctan \left( -r \left( \sin \left( 
\theta_{1} \left( t \right)  \right) -\sin \left( \theta_{3} \left( t
 \right)  \right)  \right) ,-r\cos \left( \theta_{1} \left( t \right) 
 \right) -r\cos \left( \theta_{3} \left( t \right)  \right) +q_{31}
 \left( t \right) -q_{11} \left( t \right)  \right) -\theta_{2}
 \left( t \right) \end {array} \right]
\end{equation*}

\begin{align*}
   Q_D =  \left[ \begin {array}{c} \theta_{1} \left( t \right) 
\\ \noalign{\medskip}\theta_{3} \left( t \right) \\ \noalign{\medskip}
\theta_{2} \left( t \right) \\ \noalign{\medskip}l_{1} \left( t
 \right) \\ \noalign{\medskip}l_{3} \left( t \right) 
\\ \noalign{\medskip}q_{12} \left( t \right) \\ \noalign{\medskip}q_{
32} \left( t \right) \\ \noalign{\medskip}q_{31} \left( t \right) 
\end {array} \right] 
Q_I =    \left[ \begin {array}{c} q_{11} \left( t \right) \end {array}
 \right]
\end{align*}

\subsection{Solving the Constraint Equation}

$\Phi$ can be solved by representing the dependent variables as equations of the independent variable $q_{11}(t)$.
With the analytic solution it is possible to proceed to the analysis of the system.

The solution for the fixed rotation yields 

\begin{equation*}
    \left[ \begin {array}{c} \theta_{1} \left( t \right) =\arctan \left( 
1/12\,\sqrt {21-48\, \left( q_{11} \left( t \right)  \right) ^{2}-72\,
q_{11} \left( t \right) }\sqrt {3},-q_{11} \left( t \right) -3/4
 \right) \\ \noalign{\medskip}\theta_{3} \left( t \right) =\arctan
 \left( 1/3\,\sqrt {21-48\, \left( q_{11} \left( t \right)  \right) ^{
2}-72\,q_{11} \left( t \right) }\sqrt {3},\sqrt {16\, \left( q_{11}
 \left( t \right)  \right) ^{2}+24\,q_{11} \left( t \right) +9}
 \right) \\ \noalign{\medskip}\theta_{2} \left( t \right) =0
\\ \noalign{\medskip}l_{1} \left( t \right) =1/6\,\sqrt {21-48\,
 \left( q_{11} \left( t \right)  \right) ^{2}-72\,q_{11} \left( t
 \right) }\\ \noalign{\medskip}l_{3} \left( t \right) =1/6\,\sqrt {21-
48\, \left( q_{11} \left( t \right)  \right) ^{2}-72\,q_{11} \left( t
 \right) }\\ \noalign{\medskip}q_{12} \left( t \right) =1/12\,\sqrt {
21-48\, \left( q_{11} \left( t \right)  \right) ^{2}-72\,q_{11}
 \left( t \right) }-3/4\\ \noalign{\medskip}q_{32} \left( t \right) =-
1/12\,\sqrt {21-48\, \left( q_{11} \left( t \right)  \right) ^{2}-72\,
q_{11} \left( t \right) }+3/4\\ \noalign{\medskip}q_{31} \left( t
 \right) =1/4\,\sqrt {16\, \left( q_{11} \left( t \right)  \right) ^{2
}+24\,q_{11} \left( t \right) +9}+3/4\end {array} \right] 
\end{equation*}
